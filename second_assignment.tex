\documentclass{article}
\usepackage{amssymb}
\usepackage{amsmath}

\def\ojoin{\setbox0=\hbox{$\bowtie$}%
  \rule[-.02ex]{.25em}{.4pt}\llap{\rule[\ht0]{.25em}{.4pt}}}
\def\leftouterjoin{\mathbin{\ojoin\mkern-5.8mu\bowtie}}
\def\rightouterjoin{\mathbin{\bowtie\mkern-5.8mu\ojoin}}
\def\fullouterjoin{\mathbin{\ojoin\mkern-5.8mu\bowtie\mkern-5.8mu\ojoin}}

\begin{document}
Parley Pacheco Martins 1484000 - Assignment 2 \\

1 - c) 
	\begin{equation}
	\begin{split}
		sbc\_employees \leftarrow \sigma_{company\_name = "Small\ Bank\ Corporation"} (works)\\
		\Pi_{person\_name} (\sigma_{works.salary > sbc\_employees.salary}(works \bowtie sbc\_emplyees))
	\end{split}
	\end{equation}

2 - a)
	\begin{equation}
		\Pi_{person\_name} (\sigma_{company\_name = "First\ Bank\ Corporation"})
	\end{equation}

	c)
	\begin{equation}
	\begin{split}
		jobs \leftarrow employee \bowtie works \\
		fbc \leftarrow \sigma_{company\_name = "First\ Bank\ Corporation", salary > 10,000}(jobs) \\
		\Pi_{person\_name, street, city} (fbc)
	\end{split}
	\end{equation}

3 - \begin{equation}
		\Pi_{customer\_name, customer\_city}(borrower \bowtie customer)
	\end{equation}

a) Jackson does not appear in the results because he is not in the customer relation (as seen in 
Figure 2.4). When we include the attribute \textit{city} in our projection, we remove Jackson from our results. \\

b) I would make the attribute \textit{customer\_name} in the borrower relation a foreign key, forcing any borrower to be a bank customer. \\

c)\begin{equation}
		\Pi_{customer\_name, customer\_city}(borrower \leftouterjoin customer)
	\end{equation} 


4 a)
	\begin{equation}
		\Pi_{account\_number} (G_{\textbf{count}(account\_number) > 1} (depositor))
	\end{equation}


5 a)
	\begin{equation}
	\begin{split}
		\{t\ |\ \exists\ s \in works(t[person\_name] = s[person\_name]\ \wedge \\
		 s[company\_name] = "First\ Bank\ Corporation"\}
	\end{split}
	\end{equation}

	c)
	\begin{equation}
	\begin{split}
		\{t\ |\ \exists\ s \in works(t[person\_name] = s[person\_name]\ \wedge \\
		 s[company\_name] = "First\ Bank\ Corporation"\ \wedge \\
		 s[salary] > 10,000)\ \wedge \\
		  \exists\ u \in employee(u[person\_name] = s[person\_name]\ \wedge \\
		   t[street] = u[street] \wedge t[city] = u[city])\}
	\end{split}
	\end{equation}

6 - \\
a)
SELECT s.stuName, s.program, r.courseId \\
FROM student AS s, register AS r \\
WHERE s.studentId = r.studentid and courseId like "CSC*" \\
ORDER BY s.stuName, r.courseId DESC; \\

b)
SELECT DISTINCT s.studentId, s.stuName \\
FROM student AS s, studentArea AS a, register AS r, offering AS o \\
WHERE a.depth = "major" and a.area="CSC" and s.studentId = a.studentId and r.studentId = s.studentId and r.courseId = o.courseId and r.sectId = o.sectId and ((r.grade \textless\ 2 and o.term = "Fall") or (r.approval = No and o.term="Winter")); \\

c)
SELECT s.studentId, s.stuName, avg(r.grade) AS gradeAvg \\
FROM student AS s, register AS r, offering AS o \\
WHERE s.studentId=r.studentId AND r.courseId=o.courseId AND r.sectId=o.sectId AND o.term="Fall" \\
GROUP BY s.studentId, s.stuName \\
HAVING avg(r.grade) \textgreater \ 2.5 \\
ORDER BY s.stuName; \\

d)
SELECT p.profName \\
FROM prof AS p, teach AS t, register AS r \\
WHERE p.profId = t.profId and t.courseId = r.courseId and t.sectId = r.sectId and r.approval = Yes \\
GROUP BY p.profName \\
HAVING count(r.studentId) \textgreater \ 3; \\

e)SELECT s.studentId, s.stuName \\
FROM student AS s \\
WHERE 2 \textless \ all(select grade from register AS r, offering AS o \\
where s.studentId = r.studentId \\
and r.courseId = o.courseId and r.sectId = o.sectId and o.term = "Fall") \\
GROUP BY s.studentId, s.stuName \\
ORDER BY s.stuName; \\

f)
SELECT DISTINCT p.profName \\
FROM prof AS p \\
WHERE (((p.profName) Not In (select p.profName \\
FROM prof AS p, teach AS t, offering AS o, location AS l \\
WHERE p.profId = t.profId and t.sectid = o.sectId and t.courseId = o.courseId and o.term="Winter" and l.courseId = o.courseId and l.sectId = o.sectId and l.time like "*F*"))) \\
ORDER BY p.profName; \\

g)(I know this isn't the most efficient way, but I couldn't use that temporary table that a regular sql implementation has)\\
SELECT s.studentId, s.stuName, r.grade\\
FROM student AS s, register as r \\
WHERE r.courseId = "PHY102" and s.studentId = r.studentId and r.grade =(select min(grade) from register  where courseId = "PHY102");\\

h)
SELECT  o.courseId, o.sectId, count(r.studentId) AS qty\_students \\
FROM register r right join offering o on r.courseId = o.courseId and r.sectId = o.sectId \\
GROUP BY o.courseId, o.sectId \\
ORDER BY o.courseId asc, o.sectId asc; \\

i)

j)




\end{document}
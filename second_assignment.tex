\documentclass{article}
\usepackage{amssymb}

\def\ojoin{\setbox0=\hbox{$\bowtie$}%
  \rule[-.02ex]{.25em}{.4pt}\llap{\rule[\ht0]{.25em}{.4pt}}}
\def\leftouterjoin{\mathbin{\ojoin\mkern-5.8mu\bowtie}}
\def\rightouterjoin{\mathbin{\bowtie\mkern-5.8mu\ojoin}}
\def\fullouterjoin{\mathbin{\ojoin\mkern-5.8mu\bowtie\mkern-5.8mu\ojoin}}

\begin{document}
2.1 - a) 
	\begin{equation}
		jobs \leftarrow employee \bowtie works 
	\end{equation}
	\begin{equation}
		\Pi_{person\_name} (\sigma_{company_name = "Small\ Bank\ Corporation"})
	\end{equation}

2.5 - a)
	\begin{equation}
		\Pi_{person\_name} (\sigma_{company_name = "First\ Bank\ Corporation"})
	\end{equation}

	c)
	\begin{equation}
		jobs \leftarrow employee \bowtie works 
	\end{equation}
	\begin{equation}
		\sigma_{company\_name = "First\ Bank\ Corporation", salary > 10,000}(jobs)
	\end{equation}

2.6 - \begin{equation}
		\Pi_{customer\_name, customer\_city}(borrower \bowtie customer)
	\end{equation}

a) Jackson does not appear in the results because he is not in the customer relation (as seen in 
Figure 2.4). When we include the attribute \textit{city} in our projection, we remove Jackson from our results. \\

b) I would make the attribute \textit{customer\_name} in the borrower relation a foreign key, forcing any borrower to be a bank customer. \\

c)\begin{equation}
		\Pi_{customer\_name, customer\_city}(borrower \leftouterjoin customer)
	\end{equation}


2.8 a)
	\begin{equation}
		\Pi_{account\_number} (G_{\textbf{count}(account\_number) > 1} (account))
	\end{equation}

	b)
	\begin{equation}
		\Pi_{account\_number} (G_{\textbf{count}(account\_number) > 1} (account))
	\end{equation}



\end{document}